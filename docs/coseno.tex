\documentclass[a4paper,10pt]{article}

\usepackage[spanish]{babel}
\usepackage[utf8]{inputenc}
\usepackage{latexsym,amsfonts,amssymb,amstext,amsthm,float,amsmath}
\usepackage{doc}
\usepackage[dvips]{epsfig}

%%%%%%%%%%%%%%%%%%%%%%%%%%%%%%%%%%%%%%%%%%%%%%%%%%%%%%%%%%%%%%%%%%%%%%
%123456789012345678901234567890123456789012345678901234567890123456789
%%%%%%%%%%%%%%%%%%%%%%%%%%%%%%%%%%%%%%%%%%%%%%%%%%%%%%%%%%%%%%%%%%%%%%
%\textheight    29cm
%\textwidth     15cm
%\topmargin     -4cm
%\oddsidemargin  5mm

%%%%%%%%%%%%%%%%%%%%%%%%%%%%%%%%%%%%%%%%%%%%%%%%%%%%%%%%%%%%%%%%%%%%%%

\begin{document}
\title{Búsqueda de raíces de  $cos({\pi}x)$ mediante el Método de Newton}
\author{Alba Crespo Pérez, Raquel Espino Mantas y Robbert Jozef Michiels}
\date{12 de mayo de 2014}

\maketitle

\newpage

\begin{abstract}
Elaborar un informe científico en \LaTeX{} del cálculo de las raíces $cos({\pi}x)$ utilizando el Método de Newton. Para ello se formula un código Python que calcule dichas raíces y se hará una presentación en BEAMER como ayuda para la presentación.
\end{abstract}
\newpage
\section{Motivación y Objetivos}
Este proyecto se ha realizado por ser parte de la evaluación continua de la asignatura de Técnicas Experimentales.
\begin{enumerate}
  \item
    Objetivos principales:
      \begin{enumerate}
        \item
          Aprender a programar en el lenguaje Python.
        \item
          Saber elaborar un informe de forma científica en el programa \LaTeX 
        \item
          Transmitir la información oralmente con la ayuda de una presentación BEAMER. 
       \end{enumerate}
  \item
    Objetivo específicos:
       \begin{enumerate}
          \item
           Empleo del método de Newton aplicado a la función $cos({\pi}x)$ en una computadora.
       \end{enumerate}
\end{enumerate}

\newpage

\section{Fundamentación teórica}
El método de Newton (conocido también como el método de Newton-Raphson o el método de Newton-Fourier) es un algoritmo eficiente para encontrar aproximaciones de los ceros o raíces de una función real. También puede ser usado para encontrar el máximo o mínimo de una función, encontrando los ceros de su primera derivada.

\subsection{Descripción del método}
El llamado MÉTODO DE NEWTON es un procedimiento iterativo para calcular valores aproximados de una raiz o un cero de la ecuación f (x) = 0, partiendo de un punto conocido y cercano a la raiz buscada. 
Para utilizar el método de Newton se ha de comenzar la iteración con un valor razonablemente cercano al cero (denominado punto de arranque o valor supuesto). La relativa cercanía del punto inicial a la raíz depende mucho de la naturaleza de la propia función; si ésta presenta múltiples puntos de inflexión o pendientes grandes en el entorno de la raíz, entonces las probabilidades de que el algoritmo diverja aumentan, lo cual exige seleccionar un valor supuesto cercano a la raíz. Una vez que se ha hecho esto, el método linealiza la función por la recta tangente en ese valor supuesto. La abscisa en el origen de dicha recta será, según el método, una mejor aproximación de la raíz que el valor anterior. Se realizarán sucesivas iteraciones hasta que el método haya convergido lo suficiente.
También es posible que no encuentre solución si el valor inicial que se le introduce al programa es un máximo o un mínimo de la función.
\subsection{Fórmula del método}
Sea f : [a, b], R función derivable definida en el intervalo real [a, b]. Empezamos con un valor inicial $x_0$ y definimos para cada número natural n

   $$ x_{n+1} = x_n - \frac{f(x_n)}{f'(x_n)}.$$

Donde f ' denota la derivada de f.

\end{document}