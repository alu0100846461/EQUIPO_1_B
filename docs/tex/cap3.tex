%%%%%%%%%%%%%%%%%%%%%%%%%%%%%%%%%%%%%%%%%%%%%%%%%%%%%%%%%%%%%%%%%%%%%%%%%%%%%%%
% Cap�tulo 3: Procedimiento experimental 
%%%%%%%%%%%%%%%%%%%%%%%%%%%%%%%%%%%%%%%%%%%%%%%%%%%%%%%%%%%%%%%%%%%%%%%%%%%%%%%

%++++++++++++++++++++++++++++++++++++++++++++++++++++++++++++++++++++++++++++++
\section{Descripci�n de los experimentos}
\label{3:sec:1}

Para llevar a cabo el c�lculo de las ra�ces de la funci�n objeto de estudio,
se ha realizado la implementaci�n de un algoritmo recursivo basado en el m�todo
de Newton-Raphson (ver ap�ndice A). Ejecutando en repetidas ocasiones el fichero
resultante y utilizando distintos valores como par�metros iniciales del m�todo,
ha sido posible observar los efectos producidos por la variabilidad de �stos 
sobre la soluci�n final generada por el algoritmo.

Por otra parte, con el fin de cumplir los objetivos espec�ficos de esta investigaci�n
(ver cap�tulo 1), hemos recurrido a la elaboraci�n de dos programas Python: el primero 
de ellos (ap�ndice 2A) est� destinado a efectuar un recuento de las iteraciones
requeridas para el c�lculo de una ra�z en funci�n del error absoluto del valor
proporcionado al m�todo como estimaci�n inicial, almacenando las cantidades obtenidas
en un fichero para su posterior lectura, mientras que el segundo (ap�ndice 2B)
posibilita la elaboraci�n de una gr�fica ilustrativa tomando como base la informaci�n
contenida en el fichero generado con anterioridad.

%++++++++++++++++++++++++++++++++++++++++++++++++++++++++++++++++++++++++++++++
\section{Descripci�n del material}
\label{3:sec:2}

Como soporte f�sico requerido para la puesta en pr�ctica de los experimentos,
ha sido utilizada una computadora con las especificaciones de hardware que se 
detallan a continuaci�n:

\begin{itemize}
    \item
      \textbf{Tipo de CPU:} Pentium(R) Dual-Core CPU, GenuineIntel, T4500.
    \item
      \textbf{Velocidad de la CPU:} 1200 Hz.
    \item
      \textbf{Tama�o del cach�:} 1024 KB.
    \item
      \textbf{Memoria RAM:} 8 GB.
\end{itemize}

En lo referente a las caracter�sticas y versiones del software empleado, cabe 
se�alar la siguente informaci�n:

\begin{itemize}
    \item
      \textbf{Versi�n de Python:} 2.7.3.
    \item
      \textbf{Compilador Python:} GCC 4.7.2.
    \item
      \textbf{Sistema operativo:} Linux-3.5.0-17-gen�rico con Ubuntu-12.10-quantal.
    \item
      \textbf{Fecha de creaci�n de la versi�n de Python:} Sep 26 2012 21:53:58.
\end{itemize}

%++++++++++++++++++++++++++++++++++++++++++++++++++++++++++++++++++++++++++++++
\section{Resultados obtenidos}
\label{3:sec:3}

Subsecci�n 3

%------------------------------------------------------------------------------
%%--------------------------------------------------------------------------
\begin{table}[!ht]
\begin{center}
\begin{tabular}{|c|c|} \hline 
\textbf{Tiempo  } & \textbf{Velocidad} \\ 
\textbf{($\pm$ 0.001 s)} & \textbf{($\pm$ 0.1 m/s)} \\ \hline \hline
1.234 &
67.8
\\
\hline

2.345 &
78.9
\\
\hline

3.456 &
89.1
\\
\hline

4.567 &
91.2
\\
\hline

\end{tabular}
\end{center}
\caption{Resultados experimentales de tiempo (s) y velocidad (m/s)}
\label{tab:1}
\end{table}

\begin{table}
\end{table}
\begin{table}{[!ht]}
 \begin{center}
  \begin{tabular}{|l|c|c|}
   \hline
   Nombre & Edad & Nota \\ \hline
   Pepe   & 24   & 10   \\ \hline
   Juan   & 19   & 8    \\ \hline
   Luis   & 21   & 9    \\ \hline
\end{tabular}
\end{center}
\caption{Mi primer cuadro de datos}
\label{tab}
\end{table}


%------------------------------------------------------------------------------

%++++++++++++++++++++++++++++++++++++++++++++++++++++++++++++++++++++++++++++++
\section{An�lisis de los resultados}
\label{3:sec:4}

Subsecci�n 4

