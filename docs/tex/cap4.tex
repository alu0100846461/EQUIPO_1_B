%%%%%%%%%%%%%%%%%%%%%%%%%%%%%%%%%%%%%%%%%%%%%%%%%%%%%%%%%%%%%%%%%%%%%%%%%%%%%%%
% Capítulo 4: Conclusiones y Trabajos Futuros 
%%%%%%%%%%%%%%%%%%%%%%%%%%%%%%%%%%%%%%%%%%%%%%%%%%%%%%%%%%%%%%%%%%%%%%%%%%%%%%%

\setlength{\parskip}{2mm}

En esta secci\'on se facilita un resumen de las principales conclusiones alcanzadas
tras la finalizaci\'on y an\'alisis de los experimentos llevados a cabo durante la
realizaci\'on del presente trabajo:

\begin{enumerate}
    \item
      Las ra\'ices de la funci\'on $f(x) = cos (\pi x)$ vienen dadas de la siguiente forma:
      \begin{center}
      $x = \frac{1}{2} + k, \ k \in \mathbb{Z} \iff x = ..., -2.5, -1,5, -0.5, 0.5, 1.5, 2.5, ...$
      \end{center}
    \item
      Para el caso de la funci\'on de estudio, la convergencia del algoritmo de Newton-Raphson
      se encuentra pr\'acticamente garantizada debido a la reducida separaci\'on entre ra\'ices y 
      a su car\'acter de periodicidad, salvo en el caso de seleccionar un valor inicial coincidente
      con un extremo relativo de la funci\'on.
    \item
      La eficiencia del algoritmo implementado resulta ser considerablemente elevada, al requerir
      \'infimos tiempos de CPU para la determinaci\'on de una ra\'iz y no superar en ning\'un caso el 
      margen de cuatro iteraciones hasta obtener la soluci\'on exacta.
    \item
      El m\'etodo de Newton-Raphson demuestra caracterizarse por una notable velocidad de convergencia
      para la funci\'on considerada, llegando a proporcionar hasta cinco cifras decimales correctas a 
      lo largo de una \'unica iteraci\'on.
    \item
      Las alteraciones en los par\'ametros iniciales tan s\'olo afectan al tiempo de uso de CPU a trav\'es
      de la modificaci\'on de la cantidad total de invocaciones recursivas requeridas para alcanzar
      una ra\'iz v\'alida, si bien el reducido valor de dicho tiempo origina que las diferencias globales
      resulten despreciables tanto desde el punto de vista del usuario como en un sentido computacional.
\end{enumerate}


