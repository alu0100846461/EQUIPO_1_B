%%%%%%%%%%%%%%%%%%%%%%%%%%%%%%%%%%%%%%%%%%%%%%%%%%%%%%%%%%%%%%%%%%%%%%%%%%%%%%%%%%%%%%%
%           APÉNDICE 1
%%%%%%%%%%%%%%%%%%%%%%%%%%%%%%%%%%%%%%%%%%%%%%%%%%%%%%%%%%%%%%%%%%%%%%%%%%%%%%%%%%%%%%%

\begin{center}
\begin{footnotesize}

\begin{verbatim}
#######################################################################################
# Fichero newton.py
#######################################################################################
#
#  AUTORES: Alba Crespo Perez, Raquel Espino Mantas y Robbert Jozef Michiels
#   
#  FECHA: 5 de mayo de 2014
#
#  DESCRIPCION: Este codigo Python nos permite calcular las raices de la funcion
#  f(x) = cos (pi * x), mediante la aplicacion del metodo de Newton-Raphson. Como
#  parametros de inicio se solicita al usuario una estimacion de la raiz, el margen
#  de tolerancia permitido y una cota maxima de iteraciones.
#
#######################################################################################

#!encoding: UTF-8

from math import cos
from math import sin
PI = 3.141592653589793116

def f(x):
    return cos (PI * x)

def df(x):
    return - PI * sin (PI * x)

def newton (g, tol, nmax, it):
    if (it < nmax):
        if (df(g) != 0):
            g = g - (f(g)/df(g))
        else:
            return 1e7
        if (abs(f(g)) > tol):
            g = newton (g, tol, nmax, it)
            it = it + 1
        if (abs(f(g)) < tol):
            return g
    else:
        return 1e6

g = float(raw_input("\nProporcione una estimacion para iniciar el calculo: "))
tol = float(raw_input("Introduzca el margen de tolerancia: "))
nmax = int(raw_input("Indique la cantidad maxima de iteraciones: "))
it = 0
sol = newton (g, tol, nmax, it)
if (sol == 1e6):
    print "\n\tLo sentimos, no hemos localizado ninguna raiz \n\ttras alcanzar el maximo
            de iteraciones permitidas."
    print "\nIntentelo de nuevo proporcionando una mejor estimacion como inicio del método,\n
            o bien incrementando la cota de iteraciones.\n"
elif (sol == 1e7):
    print "\n\tHemos alcanzado una anulacion de la derivada durante la ejecucion, \n\tpor lo 
           que el metodo no es aplicable para los valores aportados.\n"
    print "Intentelo de nuevo modificando los parametros iniciales.\n"
else:
    print "\nRaiz encontrada para la funcion:", sol, "\n"

\end{verbatim}

\end{footnotesize}
\end{center}

