%%%%%%%%%%%%%%%%%%%%%%%%%%%%%%%%%%%%%%%%%%%%%%%%%%%%%%%%%%%%%%%%%%%%%%%%%%%%%%%
% Cap�tulo 1: Motivaci�n y Objetivos 
%%%%%%%%%%%%%%%%%%%%%%%%%%%%%%%%%%%%%%%%%%%%%%%%%%%%%%%%%%%%%%%%%%%%%%%%%%%%%%%

%---------------------------------------------------------------------------------

\setlength{\parskip}{2mm}

El objetivo principal del presente trabajo reside en la implementaci�n, mediante
el uso del lenguaje de programaci�n Python, de un algoritmo basado en el m�todo
de Newton - Raphson que nos permita llevar a cabo el c�lculo de las ra�ces de la
funci�n $f(x) = cos(\pi x)$.

Adem�s, se pretende lograr la consecuci�n de los siguientes objetivos espec�ficos:

\begin{itemize}
    \item
      Efectuar un an�lisis num�rico y gr�fico de la evoluci�n en el n�mero de 
      iteraciones requeridas por este m�todo para la obtenci�n de una ra�z, en
      funci�n del error absoluto de la estimaci�n inicial tomada como punto de 
      partida del m�todo respecto a la soluci�n real determinada tras su aplicaci�n.
    \item
      Analizar el costo computacional, en t�rminos de tiempo de uso de CPU, asociado 
      a la ejecuci�n del algoritmo implementado, as� como su variabilidad y sensibilidad
      ante modificaciones en los par�metros iniciales.
\end{itemize}

%---------------------------------------------------------------------------------

