\section{Recuento de iteraciones}
\label{Apendice2:label}

\begin{center}
%%%%%%%%%%%%%%%%%%%%%%%%%%%%%%%%%%%%%%%%%%%%%%%%%%%%%%%%%%%%%%%%%%%%%%%%%%%%%%%%%%%%%%%
%           APÉNDICE 2
%%%%%%%%%%%%%%%%%%%%%%%%%%%%%%%%%%%%%%%%%%%%%%%%%%%%%%%%%%%%%%%%%%%%%%%%%%%%%%%%%%%%%%%

\begin{footnotesize}

\begin{verbatim}
#######################################################################################
# Fichero countiter.py
#######################################################################################
#
#  AUTORES: Alba Crespo Perez, Raquel Espino Mantas y Robbert Jozef Michiels
#   
#  FECHA: 5 de mayo de 2014
#
#  DESCRIPCION: El codigo Python que a continuacion se presenta cumple la funcion
#  de determinar la cantidad de iteraciones requeridas para la deteccion de la raiz
#  x = 0.5, tomando como estimacion inicial a los distintos valores comprendidos en
#  el intervalo [0.2, 0.5), con una diferencia de una centesima entre dos valores 
#  consecutivos. Los resultados finales seran almacenados en un fichero.
#
#######################################################################################

#!encoding: UTF-8

from math import cos
from math import sin
PI = 3.141592653589793116

def f(x):
    return cos (PI * x)

def df(x):
    return - PI * sin (PI * x)

def newton (g, tol, nmax, it, name):
    if (it < nmax):
        if (df(g) != 0):
            g = g - (f(g)/df(g))
        else:
            return 1e7
        if (abs(f(g)) > tol):
            g = newton (g, tol, nmax, it, name)
            it = it + 1
        if (abs(f(g)) < tol):
            fich = open(name, "a")
            fich.write(str(it) + "\n")
            fich.close()
            return g
    else:
        return 1e6

tol = float(raw_input("\nIntroduzca el margen de tolerancia: "))
nmax = int(raw_input("Indique la cantidad maxima de iteraciones: "))
name = raw_input("Especifique un nombre para el fichero de salida: ")

for i in range (20, 50):
    x = i/100.0
    sol = newton (x, tol, nmax, 0, name)
    if (sol != 1e6) and (sol != 1e7):
        fich = open(name, "a")
        fich.write (str(x) + "\n")
        fich.close()

print "\nOperacion realizada con exito.\n"

\end{verbatim}

\end{footnotesize}
\end{center}

\section{Creando figura con evoluci\'on de las iteraciones}
\label{Apendice2:label2}

\begin{center}
\begin{footnotesize}

\begin{verbatim}
#######################################################################################
# Fichero graphiter.py
#######################################################################################
#
#  AUTORES: Alba Crespo Perez, Raquel Espino Mantas y Robbert Jozef Michiels
#   
#  FECHA: 5 de mayo de 2014
#
#  DESCRIPCION: A partir de la lectura e interpretacion de la informacion contenida
#  en el fichero generado por el programa anterior, este codigo nos permite elaborar
#  una representacion grafica de los datos experimentales obtenidos para asi facilitar 
#  su comprension y visualizacion. 
#
#######################################################################################

#!encoding: UTF-8

import matplotlib.pyplot as pl
pl.rc('text', usetex=True)
pl.rc('font', family='Bookman')

name = raw_input("Introduzca el nombre del fichero para lectura: ")
f = open(name, "r")

flag = 0
X = []
Y = []

while (True):
    l = f.readline().rstrip()
    if (l == ""):
        break
    elif (l != "0") and (l != "1"):
        Y = Y + [flag-1]
        flag = 0
    else:
        flag += 1

f.close()

for i in range (20, 50):
    x0 = i/100.0
    dif = 0.5 - x0
    X = X + [dif]

pl.plot(X,Y,"b")
pl.title(r'Evolucion en el numero de iteraciones del metodo de Newton') 
pl.xlabel(r'Error absoluto de la estimacion inicial')
pl.ylabel('Cantidad de iteraciones')

pl.xlim(-0.05, 0.35)
pl.ylim(0.8, 3.2)

pl.savefig("iterevol.eps", dpi=72)
pl.show()

\end{verbatim}


\end{footnotesize}
\end{center}

\section{Estudio del tiempo de CPU}
\label{Apendice2:label3}

\begin{center}
\begin{footnotesize}

\begin{verbatim}
#######################################################################################
# Fichero CPUtime.py
#######################################################################################
#
#  AUTORES: Alba Crespo Perez, Raquel Espino Mantas y Robbert Jozef Michiels
#   
#  FECHA: 5 de mayo de 2014
#
#  DESCRIPCION: Este programa cumple la funcion de representar graficamente el tiempo
#  de CPU requerido en la ejecucion de 10^4 invocaciones al metodo de Newton-Raphson
#  tomando para ello cuatro posibles combinaciones de parametros iniciales que 
#  precisan, respectivamente, de 1, 2, 3 y 4 llamadas recursivas al algoritmo.
#
#######################################################################################

#!encoding: UTF-8

import matplotlib.pylab as pl
import numpy as np
import newton
import time

pl.rc('text', usetex=True)
pl.rc('font', family='Bookman')

g = (3.45,-5.8,-7.2,2.17)
tol = (10e-4,10e-5, 10e-7,10e-12)
nmax = (10,50,100,10)

X = []
Y = []

for i in range (0, 4):
    t0 = time.clock()
    for j in range (0, 10000):
        sol = newton.newton (g[i], tol[i], nmax[i], 0)
    tf = time.clock() - t0
    X = X + [i+1]
    Y = Y + [tf]

print Y

pl.plot(X,Y,'mp--')
pl.title(r'Costo computacional del m\'etodo de Newton') 
pl.xlabel(r'Cantidad de iteraciones (en decenas de miles)')
pl.ylabel(r'Tiempo de CPU (segundos)')

pl.xlim(0.5, 4.5)
pl.ylim(0, 0.2)

pl.savefig("CPUtime.eps", dpi=72)
pl.show()

\end{verbatim}

\end{footnotesize}
\end{center}

